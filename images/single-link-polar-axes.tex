\documentclass[tikz]{standalone}
\usetikzlibrary{calc,patterns}

\usepackage{amsmath}

\newcommand{\nvar}[2]{%
    \newlength{#1}
    \setlength{#1}{#2}
}

\begin{document}

% Define a few constants for drawing
\nvar{\dg}{0.3cm}
\def\dw{0.25}\def\dh{0.5}
\nvar{\ddx}{1.5cm}

% Define commands for links, joints and such
\def\link{\draw [double distance=1.5mm, very thick] (0,0)--}
\def\joint{%
    \filldraw [fill=white] (0,0) circle (5pt);
    \fill[black] circle (2pt);
}
\def\endpoint{%
    %\filldraw [fill=white] (0,0) circle (5pt);
    \fill[black] circle (2pt);
}
\def\grip{%
    \draw[ultra thick](0cm,\dg)--(0cm,-\dg);
    \fill (0cm, 0.5\dg)+(0cm,1.5pt) -- +(0.6\dg,0cm) -- +(0pt,-1.5pt);
    \fill (0cm, -0.5\dg)+(0cm,1.5pt) -- +(0.6\dg,0cm) -- +(0pt,-1.5pt);
}
\def\robotbase{%
    \draw[rounded corners=8pt] (-\dw,-\dh)-- (-\dw, 0) --
        (0,\dh)--(\dw,0)--(\dw,-\dh);
    \draw (-0.5,-\dh)-- (0.5,-\dh);
    \fill[pattern=north east lines,] (-0.5,-1) rectangle (0.5,-\dh);
}

% Draw an angle annotation
% Input:
%   #1 Angle
%   #2 Label
% Example:
%   \angann{30}{$\theta_1$}
\newcommand{\angann}[2]{%
    \begin{scope}[]
    \draw [dashed, ] (0,0) -- (1.2\ddx,0pt);
    \draw [->, shorten >=3.5pt] (\ddx,0pt) arc (0:#1:\ddx);
    % Unfortunately automatic node placement on an arc is not supported yet.
    % We therefore have to compute an appropriate coordinate ourselves.
    \node at (#1/2-2:\ddx+8pt) {#2};
    \end{scope}
}

% Draw line annotation
% Input:
%   #1 Line offset (optional)
%   #2 Line angle
%   #3 Line length
%   #5 Line label
% Example:
%   \lineann[1]{30}{2}{$L_1$}
\newcommand{\lineann}[4][0.5]{%
    \begin{scope}[rotate=#2,inner sep=2pt]
        \draw[dashed, ] (0,0) -- +(0,#1)
            node [coordinate, near end] (a) {};
        \draw[dashed, ] (#3,0) -- +(0,#1)
            node [coordinate, near end] (b) {};
        \draw[|<->|] (a) -- node[fill=white] {#4} (b);
    \end{scope}
}

% Define the kinematic parameters of the three link manipulator.
\def\thetaone{30}
\def\Lone{4}
%\def\thetatwo{30}
%\def\Ltwo{2}
%\def\thetathree{30}
%\def\Lthree{1}

%\pgfmathsetmacro{\Lout}{\Ltwo+\Lone}
%\pgfmathsetmacro{\Linner}{\Lone-\Ltwo}
%\pgfmathsetmacro{\TwoLtwo}{2*\Ltwo}

\begin{tikzpicture}[scale=1.0,]

    \robotbase
    \draw[dashed, thin] (0:\Lone)  arc (0:200:\Lone);
    \begin{scope}[rotate=90]
      \angann{\thetaone}{$\theta$}
      %\lineann[0.7]{\thetaone}{\Lone}{$l$}
    \link(\thetaone:\Lone);
    \joint
      \draw[->, thin] (0,0) -- node [at end, anchor=west] {$x$} (2,0);
      \draw[->, thin] (0,0) -- node [at end, below] {$y$} (0,2);
     \begin{scope}[shift=(\thetaone:\Lone), rotate=\thetaone]
      \endpoint
      \draw[->, thin] (0,0) -- node [at end, anchor=west] {$e_{r} = \begin{bmatrix} \cos \theta\\ \sin\theta \end{bmatrix}$} (1.5,0);
      \draw[->, thin] (0,0) -- node [at end, anchor=east] {$e_{\theta} = \begin{bmatrix} -\sin \theta\\ \cos\theta \end{bmatrix}$} (0,1.5);
    \end{scope}
  \end{scope}


\end{tikzpicture}

\end{document}
